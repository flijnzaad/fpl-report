\section{Conclusion}\label{sec:conclusion}

In conclusion, the AC-3 algorithm is a fast way to obtain an arc consistent CSP. Moreover, arc consistency is a nice way to simplify and reduce a CSP, and in some cases even solve it.
Haskell is a suitable language to represent CSPs and implement the AC-3 algorithm.

\subsection{Further improvements and research}\label{sec:further}

Since the AC-3 algorithm is not able to solve all CSPs (for example, some sudokus it was unable to solve), this program could be augmented with other algorithms.
One of those is the PC-2 algorithm, to enforce path-consistency for any CSP. Path consistency is a stronger notion of consistency: it ``tightens the binary constraints by using implicit constraints that are inferred by looking at triples of variables'' \cite[p.~210]{AIMA}.
Taking also this notion of consistency into account and maybe even for example $k$-consistency, the program could come closer to solving more CSPs, albeit at the cost of computational power.

In the specific case of analysing images of 3D objects, we could consider more elaborate situations. For example cases involving shading as suggested in \cite{winston1992}, or maybe vertices with more than three adjacent faces.
We are confident that a correct expansion of the approach we have taken can solve many of these similar problems.